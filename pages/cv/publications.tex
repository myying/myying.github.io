\begin{enumerate}
\item \textbf{Ying, Y.}, : Introducing NEDAS: a Light-weight and Scalable Python Solution for Ensemble Data Assimilation. 
\textit{J. Adv. Model. Earth Sys.}, in review.

\item \textbf{Ying, Y.}, S. Leroux, A. Korosov, E. Ólason, P. Rampal, : Predictability of sea ice linear kinematic features estimated from neXtSIM ensemble forecasts. 
in prep.

\item Wang, Y., F. Counillon, \textbf{Y. Ying}, S. Barthélémy, and G. Evensen, 2025: Improving ocean reanalysis with the offline ensemble Kalman smoother. 
\textit{Tellus A}, 77(1), 185-198. 
\href{https://doi.org/10.16993/tellusa.4087}{doi:10.16993/tellusa.4087}.

\item Korosov, A., \textbf{Y. Ying}, and E. Ólason, 2025: Tuning parameters of a sea ice model using machine learning. 
\textit{Geosc. Model Dev.}, 18, 885-904. 
\href{https://doi.org/10.5194/gmd-18-885-2025}{doi:10.5194/gmd-18-885-2025}.

\item Kay, J., T. M. Weckwerth, D. D. Turner. G. Romine, and \textbf{Y. Ying}, : Impact of assimilating thermodynamic and kinematic profiles on a convection initiation forecast. 
\textit{Mon. Wea. Rev.}, , accepted. 
\href{https://doi.org/10.1175/MWR-D-24-0233.1}{doi:10.1175/MWR-D-24-0233.1}.

\item \textbf{Ying, Y.}, J. L. Anderson, and L. Bertino, 2023: Improving vortex position accuracy with a new multiscale alignment ensemble filter. 
\textit{Mon. Wea. Rev.}, 151, 1387-1405. 
\href{https://doi.org/10.1175/MWR-D-22-0140.1}{doi:10.1175/MWR-D-22-0140.1}.

\item Korosov, A., P. Rampal, \textbf{Y. Ying}, E. Ólason, and T. Williams, 2023: Towards improving short-term sea ice predictability using deformation observations. 
\textit{The Cryosphere}, 17, 4223-4240. 
\href{https://doi.org/10.5194/tc-17-4223-2023}{doi:10.5194/tc-17-4223-2023}.

\item Tao, D., P. J. van Leeuwen, M. Bell, and \textbf{Y. Ying}, 2022: Dynamics and predictability of tropical cyclone rapid intensification in ensemble simulations of Hurricane Patricia (2015). 
\textit{J. Geophys. Res. Atmos.}, 127, e2021JD036079. 
\href{https://doi.org/10.1029/2021JD036079}{doi:10.1029/2021JD036079}.

\item \textbf{Ying, Y.}, 2020: Assimilating observations with spatially correlated errors using a serial ensemble filter with a multiscale approach. 
\textit{Mon. Wea. Rev.}, 148, 3397-3412. 
\href{https://doi.org/10.1175/MWR-D-19-0387.1}{doi:10.1175/MWR-D-19-0387.1}.

\item \textbf{Ying, Y.}, 2019: A multiscale alignment method for ensemble filtering with displacement errors. 
\textit{Mon. Wea. Rev.}, 147, 4553-4565. 
\href{https://doi.org/10.1175/MWR-D-19-0170.1}{doi:10.1175/MWR-D-19-0170.1}.

\item \textbf{Ying, Y.}, and F. Zhang, 2018: Potentials in improving predictability of multiscale tropical weather systems evaluated through ensemble assimilation of simulated satellite-based observations. 
\textit{J. Atmos. Sci.}, 75, 1675-1698. 
\href{https://doi.org/10.1175/JAS-D-17-0245.1}{doi:10.1175/JAS-D-17-0245.1}.

\item \textbf{Ying, Y.}, F. Zhang, and J. L. Anderson, 2018: On the selection of localization radius in ensemble filtering for multiscale quasi-geostrophic dynamics. 
\textit{Mon. Wea. Rev.}, 146, 543–560. 
\href{https://doi.org/10.1175/MWR-D-17-0336.1}{doi:10.1175/MWR-D-17-0336.1}.

\item \textbf{Ying, Y.}, and F. Zhang, 2017: Practical and intrinsic predictability of multi-scale weather and convectively-coupled equatorial waves during the active phase of an MJO. 
\textit{J. Atmos. Sci.}, 74, 3771-3785. 
\href{https://doi.org/10.1175/JAS-D-17-0157.1}{doi:10.1175/JAS-D-17-0157.1}.

\item \textbf{Ying, Y.}, and F. Zhang, 2015: An adaptive covariance relaxation method for ensemble data assimilation. 
\textit{Quart. J. Roy. Meteor. Soc.}, 141, 2898-2906. 
\href{https://doi.org/10.1002/qj.2576}{doi:10.1002/qj.2576}.

\item Wang, S., A. H. Sobel, F. Zhang, Y. Sun, \textbf{Y. Ying}, and L. Zhou, 2015: Regional simulation of the October and November MJO events observed during the CINDY/DYNAMO field campaign at gray zone resolution. 
\textit{J. Climate}, 28, 2097-2119. 
\href{https://doi.org/10.1175/JCLI-D-14-00294.1}{doi:10.1175/JCLI-D-14-00294.1}.

\item Hu, H., Q. Zhang, B. Xie, \textbf{Y. Ying}, J. Zhang, and X. Wang, 2014: Predictability of an advection fog event over North China. Part I: Sensitivity to initial condition differences. 
\textit{Mon. Wea. Rev.}, 142, 1803-1822. 
\href{https://doi.org/10.1175/MWR-D-13-00004.1}{doi:10.1175/MWR-D-13-00004.1}.

\item Zhang, J., T. Zhu, Q. Zhang, C. Li, and H. Shu, \textbf{Y. Ying}, Z. Dai, X. Wang, 2012: The impact of circulation patterns on regional transport pathways and air quality over Beijing and its surroundings. 
\textit{Atmos. Chem. Phys.}, 12, 5031-5053. 
\href{https://doi.org/10.5194/acpd-11-33465-2011}{doi:10.5194/acpd-11-33465-2011}.

\item \textbf{Ying, Y.}, and Q. Zhang, 2012: A modeling study on tropical cyclone structural changes in response to ambient moisture variations. 
\textit{J. Meteorol. Soc. Japan}, 90, 755-770. 
\href{https://doi.org/10.2151/jmsj.2012-512}{doi:10.2151/jmsj.2012-512}.

\item Du, Y., Q. Zhang, \textbf{Y. Ying}, and Y. Yang, 2012: Characteristics of low-level jets in Shanghai during the 2008-2009 warm seasons as inferred from wind profiler radar data. 
\textit{J. Meteorol. Soc. Japan}, 90, 891-903. 
\href{https://doi.org/10.2151/jmsj.2012-603}{doi:10.2151/jmsj.2012-603}.

\item Xie, B., Q. Zhang, and \textbf{Y. Ying}, 2011: Trends in precipitable water and relative humidity in China: 1979-2005. 
\textit{J. Applied Meteorol. Climatol.}, 50, 1985-1994. 
\href{https://doi.org/10.1175/2011JAMC2446.1}{doi:10.1175/2011JAMC2446.1}.

\end{enumerate}
