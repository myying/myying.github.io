\documentclass{article}

\usepackage[utf8]{inputenc}
\usepackage[full]{textcomp}
\usepackage{tgtermes}
\usepackage[T1]{fontenc}
\usepackage{enumitem}
\usepackage[a4paper,left=.9in, right=.9in, top=1.2in, bottom=1.2in]{geometry}
\usepackage{ulem}
\usepackage{url}
\usepackage[dvipsnames]{xcolor}
\usepackage{amsmath}
\usepackage{hyperref}

\pagestyle{headings}
\markright{\textbf{Yue Ying}}
\setlength\parindent{2em}
\thispagestyle{empty}
\newcommand{\cvsection}[1]{\section*{\bfseries#1}}
\newcommand{\cvsubsection}[1]{\subsection*{\itshape\uline{#1}}}

\begin{document}

% name & contact info
\begin{center}
\huge{\textbf{Yue (Michael) Ying}} \\
\vspace{10pt}
\normalsize{Researcher, Data Assimilation Group \\
Nansen Environmental and Remote Sensing Center (NERSC) \\
Jahnebakken 3, 5007 Bergen, Norway \\
E-mail: \href{mailto:yue.ying@nersc.no}{\texttt{yue.ying@nersc.no}} \\
Webpage: \url{https://myying.github.io} \\
ORCID: \href{https://orcid.org/0000-0001-9988-3488}{\texttt{0000-0001-9988-3488}} }
\end{center}

\cvsection{Education}
\noindent 2018: Ph.D. Meteorology, Computational Science (minor), Pennsylvania State University \\
\indent \textit{Dissertation}: ``Ensemble data assimilation for the analysis and prediction of multiscale tropical weather systems''. \\
\indent \textit{Advisor}: Dr. Fuqing Zhang \\

\noindent 2012: M.S. Meteorology, Peking University \\
\indent \textit{Thesis}: ``Tropical cyclone structural changes in response to ambient moisture perturbations''. \\
\indent \textit{Advisor}: Dr. Qinghong Zhang \\

\noindent 2009: B.S. Atmospheric Sciences, Peking University

\cvsection{Research Interests}
\begin{itemize}[leftmargin=2em, topsep=0pt, itemsep=0pt]
    \item Advancing data assimilation methodologies for multiscale dynamical systems
    \item Dynamics and predictability of complex systems and identifying key physical processes
    \item Improving the numerical simulation and prediction of complex dynamical systems
\end{itemize}


\cvsection{Professional Experiences}
\cvsubsection{Research}
\begin{tabular}{l l}
    2020-present & Researcher, Data Assimilation group, NERSC \\
    2018-2020 & Postdoctoral Fellow, Advanced Study Program, NCAR \\
    2012-2018 & Graduate Research Assistant, Pennsylvania State University \\
    2009-2012 & Graduate Research Assistant, Peking University \\
\end{tabular}

\cvsubsection{Teaching}
\begin{tabular}{l l}
    2021: & Guest Lecturer of Crash Course on Ensemble Data Assimilation, NERSC. \\
    2018: & Lead Instructor of Data Assimilation (Meteo 597), Pennsylvania State University. \\
    2016-2017: & Guest Lecturer of Data Assimilation (Meteo 597), Pennsylvania State University. \\
    2011: & Teaching Assistant for Computer Algorithms and Data Structure, Peking University. \\
    2011: & Guest Lecturer for Scientific Data Visualization, Peking University. \\
\end{tabular}

\cvsubsection{Others}
\begin{tabular}{l p{14cm}}
    2009-2011: & Part-time High-Performance Computer system administrator, Dept. of Atmospheric and Oceanic Sciences, Peking University. \\
\end{tabular}

\cvsection{Honors and Awards}
\begin{tabular}{l l}
    2018: & Al and Betty Blackadar Scholarship, Pennsylvania State University. \\
    2018: & Best Student Presentation, 22nd AMS Conference on IOAS-AOLS. \\
    2011: & DHC Software Co. Scholarship, Peking University. \\
\end{tabular}


\cvsection{Project Management}
\begin{tabular}{l p{8cm} p{2cm} p{3cm}}
    2018-2020: & Advancing ensemble data assimilation through adaptive methodologies for state and parameter estimation of multiscale dynamical systems & Project leader & NCAR/Advanced Study Program \\
\end{tabular}

%\cvsection{Supervision of Students}

\cvsection{Publication}
\begin{enumerate}
\item \textbf{Ying, Y.}, : Introducing NEDAS: a Light-weight and Scalable Python Solution for Ensemble Data Assimilation. 
\textit{J. Adv. Model. Earth Sys.}, in review.

\item \textbf{Ying, Y.}, S. Leroux, A. Korosov, E. Ólason, P. Rampal, : Predictability of sea ice linear kinematic features estimated from neXtSIM ensemble forecasts. 
in prep.

\item Wang, Y., F. Counillon, \textbf{Y. Ying}, S. Barthélémy, and G. Evensen, 2025: Improving ocean reanalysis with the offline ensemble Kalman smoother. 
\textit{Tellus}, , . 
\href{https://doi.org/}{doi:}.

\item Korosov, A., \textbf{Y. Ying}, and E. Ólason, 2025: Tuning parameters of a sea ice model using machine learning. 
\textit{Geosc. Model Dev.}, 18, 885-904. 
\href{https://doi.org/10.5194/gmd-18-885-2025}{doi:10.5194/gmd-18-885-2025}.

\item Kay, J., T. M. Weckwerth, D. D. Turner. G. Romine, and \textbf{Y. Ying}, : Impact of assimilating thermodynamic and kinematic profiles on a convection initiation forecast. 
\textit{Mon. Wea. Rev.}, , accepted. 
\href{https://doi.org/10.1175/MWR-D-24-0233.1}{doi:10.1175/MWR-D-24-0233.1}.

\item \textbf{Ying, Y.}, J. L. Anderson, and L. Bertino, 2023: Improving vortex position accuracy with a new multiscale alignment ensemble filter. 
\textit{Mon. Wea. Rev.}, 151, 1387-1405. 
\href{https://doi.org/10.1175/MWR-D-22-0140.1}{doi:10.1175/MWR-D-22-0140.1}.

\item Korosov, A., P. Rampal, \textbf{Y. Ying}, E. Ólason, and T. Williams, 2023: Towards improving short-term sea ice predictability using deformation observations. 
\textit{The Cryosphere}, 17, 4223-4240. 
\href{https://doi.org/10.5194/tc-17-4223-2023}{doi:10.5194/tc-17-4223-2023}.

\item Tao, D., P. J. van Leeuwen, M. Bell, and \textbf{Y. Ying}, 2022: Dynamics and predictability of tropical cyclone rapid intensification in ensemble simulations of Hurricane Patricia (2015). 
\textit{J. Geophys. Res. Atmos.}, 127, e2021JD036079. 
\href{https://doi.org/10.1029/2021JD036079}{doi:10.1029/2021JD036079}.

\item \textbf{Ying, Y.}, 2020: Assimilating observations with spatially correlated errors using a serial ensemble filter with a multiscale approach. 
\textit{Mon. Wea. Rev.}, 148, 3397-3412. 
\href{https://doi.org/10.1175/MWR-D-19-0387.1}{doi:10.1175/MWR-D-19-0387.1}.

\item \textbf{Ying, Y.}, 2019: A multiscale alignment method for ensemble filtering with displacement errors. 
\textit{Mon. Wea. Rev.}, 147, 4553-4565. 
\href{https://doi.org/10.1175/MWR-D-19-0170.1}{doi:10.1175/MWR-D-19-0170.1}.

\item \textbf{Ying, Y.}, and F. Zhang, 2018: Potentials in improving predictability of multiscale tropical weather systems evaluated through ensemble assimilation of simulated satellite-based observations. 
\textit{J. Atmos. Sci.}, 75, 1675-1698. 
\href{https://doi.org/10.1175/JAS-D-17-0245.1}{doi:10.1175/JAS-D-17-0245.1}.

\item \textbf{Ying, Y.}, F. Zhang, and J. L. Anderson, 2018: On the selection of localization radius in ensemble filtering for multiscale quasi-geostrophic dynamics. 
\textit{Mon. Wea. Rev.}, 146, 543–560. 
\href{https://doi.org/10.1175/MWR-D-17-0336.1}{doi:10.1175/MWR-D-17-0336.1}.

\item \textbf{Ying, Y.}, and F. Zhang, 2017: Practical and intrinsic predictability of multi-scale weather and convectively-coupled equatorial waves during the active phase of an MJO. 
\textit{J. Atmos. Sci.}, 74, 3771-3785. 
\href{https://doi.org/10.1175/JAS-D-17-0157.1}{doi:10.1175/JAS-D-17-0157.1}.

\item \textbf{Ying, Y.}, and F. Zhang, 2015: An adaptive covariance relaxation method for ensemble data assimilation. 
\textit{Quart. J. Roy. Meteor. Soc.}, 141, 2898-2906. 
\href{https://doi.org/10.1002/qj.2576}{doi:10.1002/qj.2576}.

\item Wang, S., A. H. Sobel, F. Zhang, Y. Sun, \textbf{Y. Ying}, and L. Zhou, 2015: Regional simulation of the October and November MJO events observed during the CINDY/DYNAMO field campaign at gray zone resolution. 
\textit{J. Climate}, 28, 2097-2119. 
\href{https://doi.org/10.1175/JCLI-D-14-00294.1}{doi:10.1175/JCLI-D-14-00294.1}.

\item Hu, H., Q. Zhang, B. Xie, \textbf{Y. Ying}, J. Zhang, and X. Wang, 2014: Predictability of an advection fog event over North China. Part I: Sensitivity to initial condition differences. 
\textit{Mon. Wea. Rev.}, 142, 1803-1822. 
\href{https://doi.org/10.1175/MWR-D-13-00004.1}{doi:10.1175/MWR-D-13-00004.1}.

\item Zhang, J., T. Zhu, Q. Zhang, C. Li, and H. Shu, \textbf{Y. Ying}, Z. Dai, X. Wang, 2012: The impact of circulation patterns on regional transport pathways and air quality over Beijing and its surroundings. 
\textit{Atmos. Chem. Phys.}, 12, 5031-5053. 
\href{https://doi.org/10.5194/acpd-11-33465-2011}{doi:10.5194/acpd-11-33465-2011}.

\item \textbf{Ying, Y.}, and Q. Zhang, 2012: A modeling study on tropical cyclone structural changes in response to ambient moisture variations. 
\textit{J. Meteorol. Soc. Japan}, 90, 755-770. 
\href{https://doi.org/10.2151/jmsj.2012-512}{doi:10.2151/jmsj.2012-512}.

\item Du, Y., Q. Zhang, \textbf{Y. Ying}, and Y. Yang, 2012: Characteristics of low-level jets in Shanghai during the 2008-2009 warm seasons as inferred from wind profiler radar data. 
\textit{J. Meteorol. Soc. Japan}, 90, 891-903. 
\href{https://doi.org/10.2151/jmsj.2012-603}{doi:10.2151/jmsj.2012-603}.

\item Xie, B., Q. Zhang, and \textbf{Y. Ying}, 2011: Trends in precipitable water and relative humidity in China: 1979-2005. 
\textit{J. Applied Meteorol. Climatol.}, 50, 1985-1994. 
\href{https://doi.org/10.1175/2011JAMC2446.1}{doi:10.1175/2011JAMC2446.1}.

\end{enumerate}


\cvsection{Conference and Seminar Presentations}
\begin{enumerate}
\item Kay, J., T. Weckwerth, G. Romine, \textbf{Y. Ying}, and D. Turner, \textit{``Impact of assimilating lower‐atmospheric wind and thermodynamic profiles on evolution of ABL structures and precipitation forecasts''}, 8th ISDA, Fort Collins,  6 Jun, 2022

\item \textbf{Ying, Y.}, \textit{``Multiscale alignment ensemble filtering technique and its application in geoscience''}, EnKF Workshop, Balestrand, 30 May, 2022
(invited)

\item \textbf{Ying, Y.}, Y. Qiang Sun, and S. Wang, \textit{``Predictability of Tropical Waves and the MJO''}, Fuqing Zhang’s Contribution to the Tropical Meteorology Community, 35th Conference on Hurricanes and Tropical Meteorology, 10 May, 2022
(invited)

\item \textbf{Ying, Y.}, \textit{``Correcting position errors in sea ice linear kinematic features: application of a multiscale alignment data assimilation approach''}, AI and Data Science for the Arctic Workshop, NTNU, 29 Sep, 2021
(invited)

\item \textbf{Ying, Y.}, J. Anderson, and L. Bertino, \textit{``A multiscale alignment method for ensemble filtering applied to hurricane and sea ice models''}, EnKF Workshop,  9 Jun, 2021

\item \textbf{Ying, Y.}, \textit{``How to handle nonlinearity in multiscale problems: pushing the frontier of data assimilation methodology''}, Penn State Meteorology Colloquium, 10 Mar, 2021

\item \textbf{Ying, Y.}, \textit{``Ensemble filtering with displacement errors''}, NERSC Seminar, 12 Feb, 2020

\item \textbf{Ying, Y.}, \textit{``Developing data assimilation algorithms for multiscale dynamical systems''}, Fudan University Guang-hua International Forum for Young Scholars, 26 Dec, 2019

\item \textbf{Ying, Y.}, \textit{``Developing data assimilation algorithms for the analysis and prediction of geophysical flows across many scales''}, MMM Seminar Series, NCAR,  6 Jun, 2019

\item \textbf{Ying, Y.}, \textit{``Developing a scale-aware scheme for the ensemble filtering of geophysical flows''}, Second ADAPT Symposium, Penn State, 16 Dec, 2018

\item \textbf{Ying, Y.}, \textit{``Developing scale-aware algorithms for the ensemble filtering of geophysical flows''}, Boulder Fluid and Thermal Sciences Seminar Series, 13 Nov, 2018

\item \textbf{Ying, Y.} and F. Zhang, \textit{``An idealized assimilation experiment of satellite-based observations for the analysis and prediction of tropical multiscale weather systems''}, 6th AMS Symposium on the JCSDA, 10 Jan, 2018

\item \textbf{Ying, Y.}, F. Zhang and J. Anderson, \textit{``On the selection of localization radius in ensemble filtering for multiscale quasi-geostrophic dynamics''}, 22nd AMS Conference on IOAS-AOLS,  9 Jan, 2018

\item \textbf{Ying, Y.} and F. Zhang, \textit{``Practical and intrinsic predictability of multiscale weather and convectively coupled equatorial waves during the active phase of an MJO''}, 6th AMS Symposium on the MJO,  8 Jan, 2018

\item \textbf{Ying, Y.} and F. Zhang, \textit{``Design of a satellite-based observing system for the analysis and prediction of multi-scale weather and convectively-coupled tropical waves using EnKF ''}, 28th WAF / 24th NWP Conference, 26 Jan, 2017

\item \textbf{Ying, Y.} and F. Zhang, \textit{``Observing system design, observation impact and predictability for Madden-Julian Oscillation and tropical weather''}, 7th EnKF Data Assimilation Workshop, 27 May, 2016

\item \textbf{Ying, Y.}, J. Poterjoy, and F. Zhang, \textit{``Comparison of hybrid four-dimensional data assimilation methods with and without an adjoint model for limited-area convection-permitting weather prediction: E4DVar vs. 4DEnVar''}, 27th WAF/ 23rd NWP Conference, 30 Jun, 2015

\item Sun, Y., \textbf{Y. Ying}, F. Zhang, S. Wang, and R. Johnson, \textit{``Equatorial 2-day waves and diurnal variations during DYNAMO: Observation vs. simulation''}, 19th AMS Conference on AOFD, 20 Jun, 2013

\item \textbf{Ying, Y.} and Q. Zhang, \textit{``A model study on tropical cyclone structural changes in response to ambient moisture variations''}, 30th AMS Conference on Hurricanes and Tropical Meteorology, 18 Apr, 2012

\item \textbf{Ying, Y.}, and Q. Zhang, \textit{``A model study on tropical cyclone motion and intensification in an asymmetric moisture field''}, 8th ICMCS, Nagoya,  8 Mar, 2011

\end{enumerate}


\cvsection{Academic Services}
\cvsubsection{Peer Reviews}
Manuscript reviewer for
\textit{Monthly Weather Review},
\textit{Quarterly Journal of the Royal Meteorological Society},
\textit{Nonlinear Processes in Geophysics},
\textit{Climate Dynamics},
and
\textit{Geoscientific Model Development}.

\cvsubsection{Organization of Meetings}
\begin{tabular}{l p{4cm} p{3cm} p{6cm}}
    2020-2022: & AMS annual meetings & IOAS-AOLS session convener & ``Advances in ensemble-based data assimilation methodologies for highly nonlinear and large-dimensional systems'' \\
\end{tabular}

\cvsection{Membership and Network}
\begin{tabular}{l l}
    Since 2021: & European Geosciences Union (EGU) \\
    Since 2017: & Chi Epsilon Pi National Meteorology Honors Society \\
    Since 2012: & American Meteorological Society (AMS) \\
    Since 2018: & American Geophysical Union (AGU) \\
\end{tabular}

\end{document}


